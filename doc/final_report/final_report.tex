\documentclass{acm_proc_article-sp}

\begin{document}

\title{LifeScope: A Multimodal Mobile Time Tracking Application}

\numberofauthors{3}
\author{
% 1st. author
\alignauthor
Luca Foschini\\
       \affaddr{University of California, Santa Barbara}\\
       \email{TODO@cs.ucsb.edu}
% 2nd. author
\alignauthor
Danny Iland\\
       \affaddr{University of California, Santa Barbara}\\
       \email{iland@cs.ucsb.edu}
% 3rd. author
\alignauthor 
Ian Whitfield\\
       \affaddr{University of California, Santa Barbara}\\
       \email{ianwhitfield@cs.ucsb.edu}
}


\maketitle
\begin{abstract}
LifeScope is a ...
% Need one reference to make it compile properly.
\cite{Lamport:LaTeX}.
\end{abstract}

% A category with the (minimum) three required fields
\category{H.5}{Information Systems}{INFORMATION INTERFACES AND PRESENTATION}
\terms{Mobile computing, context-aware, multimodal interaction}

\section{Introduction}
The modern smartphone...

%%%%%%%%%%%%%%%%%%%%%%%%%%%%%%%%%%%%%%%%%%%%%%%%%%%%%%%%%%%%%%%

\section{Related Work}
Related work includes... 
\subsection {funf}
The funf Open Sensing Framework is...
\subsection {MobiCon}
Something...
\subsection {on\{X\}}
Microsoft on\{X\} is...

%%%%%%%%%%%%%%%%%%%%%%%%%%%%%%%%%%%%%%%%%%%%%%%%%%%%%%%%%%%%%%%

\section{Approach and Implementation}
In LifeScope, we use the generalization that people go places to perform actions. For example, the user might go to the gym and work out, and then return home to work on a project, and finally go to sleep.  In LifeScope, this series of events would be modeled as two places and three actions. 

Due to its widespread adoption and rich application API, we decided to implement LifeScope as a user-installable application (``App'') for the Android smartphone operating system.

\subsection {Location Management}
Since activities take place at specific locations an important first step was to be able to accurately determine a user's current location and detect when they move to a new location. For this we decided to use Alohar...

\subsection {Application Architecture}
The core component of our implementation is the Detector object...

Internally, our application is structured as a central, always-running Service component with a number of standalone Activity components that can connect to it...

%%%%%%%%%%%%%%%%%%%%%%%%%%%%%%%%%%%%%%%%%%%%%%%%%%%%%%%%%%%%%%%

\section{Results and Assessment}
Results...
Assessment...

%%%%%%%%%%%%%%%%%%%%%%%%%%%%%%%%%%%%%%%%%%%%%%%%%%%%%%%%%%%%%%%

\section{Conclusions}
Conclusion...

%%%%%%%%%%%%%%%%%%%%%%%%%%%%%%%%%%%%%%%%%%%%%%%%%%%%%%%%%%%%%%%

\bibliographystyle{abbrv}
\bibliography{bibliography}  % sigproc.bib is the name of the Bibliography in this case

% That's all folks!
\end{document}
